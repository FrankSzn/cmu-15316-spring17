\documentclass[11pt]{article}

%%%%%%%%%%%%%%%%%%%%%%%%%%%%%%%%%%%%%%%%%%%%%%%%%%%%%%%%%%% 
% EDIT THESE TWO COMMANDS TO BE YOUR ANDREW ID AND NAME
\newcommand*{\stulogin}{}
\newcommand*{\stuname}{}
%%%%%%%%%%%%%%%%%%%%%%%%%%%%%%%%%%%%%%%%%%%%%%%%%%%%%%%%%%% 

\usepackage{algorithm2e}
\usepackage{fancyheadings}
\usepackage{amsmath}
\usepackage{amsfonts}
\usepackage{amssymb}
\usepackage{fullpage-old}
\usepackage{graphics}
\usepackage{sudoku}
\usepackage{hyperref}
\usepackage{url}
\usepackage{framed}
\usepackage{xcolor}
\usepackage{algpseudocode}
\usepackage{listings}
\usepackage{tikz}
\usepackage{xspace}

\usepackage{ragged2e}
% \usepackage{xcolor}
\usepackage{tikz}
\usepackage{listings}
% \usepackage{syntax}
\usepackage{semantic}
\usepackage{mathpartir}
\usepackage{stmaryrd}
\usepackage{amsmath}
\usepackage{xspace}
\usepackage{adjustbox}
\usepackage{xspace}
\usepackage{algpseudocode}
\usepackage{mathtools}
\usepackage{array}
\usepackage{rotating}

\definecolor{mygray}{rgb}{0.5,0.5,0.5}
\definecolor{backgray}{gray}{0.95}
\lstdefinelanguage
   [x86]{Assembler}     
   [x86masm]{Assembler}    
   {morekeywords={movl, addl, subl, cmpl}} 
\lstdefinestyle{customc}{
  belowcaptionskip=1\baselineskip,
  breaklines=true,
  language=C,
  showstringspaces=false,
  numbers=none,
  % xleftmargin=1ex,
  framexleftmargin=1ex,
  % numbersep=5pt,
  % numberstyle=\tiny\color{mygray},
  basicstyle=\footnotesize\ttfamily,
  keywordstyle=\color{blue},
  commentstyle=\itshape\color{purple!40!black},  
  stringstyle=\color{orange},
  morekeywords={output,assume,observe,input,bool,then,fun,match,in,val,list,type,of,string,unit,let,bytes,mov,imul,add,sar,shr,function,forall,nat,requires,ensures,method,returns,assert,new,array,modifies,reads,old,predicate,lemma,seq,calc,nan,var,exists,invariant,decreases,string},
  tabsize=2,
  deletestring=[b]',
  backgroundcolor=\color{gray!15},
  frame=tb
}
\lstset{escapechar=@,style=customc}

\usetikzlibrary{automata,shapes,positioning}

\renewenvironment{shaded}{%
  \def\FrameCommand{\fboxsep=\FrameSep \colorbox{shadecolor}}%
  \MakeFramed{\advance\hsize-\width \FrameRestore\FrameRestore}}%
 {\endMakeFramed}
\definecolor{shadecolor}{gray}{0.95}

\newcommand{\ex}{\ensuremath{\mathbf{EX\ }}\xspace}
\newcommand{\eg}{\ensuremath{\mathbf{EG\ }}\xspace}
\newcommand{\ef}{\ensuremath{\mathbf{EF\ }}\xspace}
\newcommand{\ax}{\ensuremath{\mathbf{AX\ }}\xspace}
\newcommand{\ag}{\ensuremath{\mathbf{AG\ }}\xspace}
\newcommand{\af}{\ensuremath{\mathbf{AF\ }}\xspace}
\newcommand{\nextt}{\ensuremath{\mathbf{X\ }}\xspace}
\newcommand{\future}{\ensuremath{\mathbf{F\ }}\xspace}
\newcommand{\always}{\ensuremath{\mathbf{G\ }}\xspace}
\newcommand{\until}{\ensuremath{\mathbf{\ U\ }}\xspace}
\newcommand{\release}{\ensuremath{\mathbf{\ R\ }}\xspace}
\newcommand{\existpath}{\ensuremath{\mathbf{E\ }}\xspace}
\newcommand{\forallpath}{\ensuremath{\mathbf{A\ }}\xspace}

\newcommand{\imp}{\ensuremath{\mathsf{Imp}}\xspace}
\newcommand{\aexp}{\ensuremath{\mathsf{AExp}}\xspace}
\newcommand{\bexp}{\ensuremath{\mathsf{BExp}}\xspace}
\newcommand{\com}{\ensuremath{\mathsf{Com}}\xspace}
\newcommand{\var}{\ensuremath{\mathsf{Var}}\xspace}
\newcommand{\strue}{\ensuremath{\mathsf{true}}\xspace}
\newcommand{\sfalse}{\ensuremath{\mathsf{false}}\xspace}
\newcommand{\true}{\ensuremath{\mathit{true}}\xspace}
\newcommand{\false}{\ensuremath{\mathit{false}}\xspace}

\newcommand*{\assignmentnumb}{0}
% ---------------------------------------------------------------------------

\lhead[{\bfseries 15-316\quad Assignment \assignmentnumb\ \ \ \stulogin}]{{\bfseries 15-316\quad Assignment \assignmentnumb\ \ \ \stulogin}}
\rhead[{\bfseries\thepage}]{{\bfseries\thepage}}
\pagestyle{fancy}
\parskip 1ex
\parindent 0mm

% -----------Danny's Macros-------------

\newcommand{\problem}[1]{\section{#1}\vspace{-1em}}
\newcounter{partnumber}
\newenvironment{parts}{\begin{list}{
      % \arabic{problemnumber}\alph{partnumber}:}{
      (\alph{partnumber})}{
      \usecounter{partnumber} \setlength{\rightmargin}{\leftmargin}}
    \setlength{\itemsep}{.0 in}}{
  \end{list} \vspace*{.0 in}}

% ---------------------------------------

\begin{document}

\centerline{Instructors: Matt Fredrikson, Jean Yang \hfill TA: Samuel Yeom} 
\vspace{0.5ex}
Due date: 1/24/2017 at 11:59pm \\
\vspace{1.5ex}
\centerline{\Large\bf Assignment \assignmentnumb}
\vspace{0.5ex}
\centerline{\Large\bf \stuname}

\paragraph{Instructions:} 

Complete all the required problems listed below. You may also submit the solutions for the optional problems if you would like our feedback.

When finished, compile your solutions to a pdf, and email a zip of the PDF and relevant code to the TA by the due date. Be sure to add your Andrew ID and full name in the \texttt{stulogin} and \texttt{stuname} macros at the top of \texttt{hw0.tex}.

% \newpage
\begin{problem}{Course startup (5 points)}

\paragraph{Part 1 (0 points)}
If you have not been added to the course's Piazza, please email the instructors (\texttt{15316-spring17-staff@cs.cmu.edu}) to be added.

\paragraph{Part 2 (5 Points)}

Email the course staff (\texttt{15316-spring17-staff}) introducing yourself! Tell us your name, your major and year, how the course fits in with your core values, what you hope to get out of the course, and a fun fact about yourself. Feel free to tell us your questions and concerns as well.

\end{problem}

\newpage
\begin{problem}{Getting Started with OCaml (0 points)}

The skeleton code for this problem is in \texttt{2/exercises.ml}.

\paragraph{Part 1 (0 points)}

Follow the instructions to install OCaml on your system:
\begin{verbatim}
http://www.ocaml.org/docs/install.html
\end{verbatim}

Install the OPAM package manager and run the following to initalize OPAM, and then to install the OUnit package for unit testing:
\begin{verbatim}
opam init
opam install ounit
\end{verbatim}

\paragraph{Part 2 (0 points)}

Open an OCaml interactive session (\texttt{ocaml}) and use it to determine the types of the following functions. What do the types mean?\footnote{Problem taken from \url{https://www.cs.rice.edu/~sc40/COMP507/Assignments/assign0.pdf}}
\begin{enumerate}
\item \texttt{let f (x,y) = x :: y}
\item \texttt{let f (g, h) = function x -> g (h x)}
\item \texttt{let f g h = function x -> g (h x)}
\item \texttt{let f x y z = if x < y then "hello" else z}
\end{enumerate}

\paragraph{Part 3 (0 points)}

Complete the code and tests from \texttt{exercises0.ml} to write the following functions:
\begin{enumerate}
\item Fibonacci.
\item List reversal.
\item A function \texttt{filter lst f} that takes a list \texttt{lst: 'a} and a function \texttt{f: 'a -> bool} and returns a list {r: 'a} of the elements in \texttt{f} satisfying \texttt{f}.
\end{enumerate}
Run the code using the following command to create the executable \texttt{exercises0}:
\begin{verbatim}
ocamlfind ocamlc -package oUnit -linkpkg -g -o exercises0 exercises0.ml 
\end{verbatim}
%%%%%%%%%%%%%%%%%%%%%%%%%%%%%%%%%%%%%%%%%%%%%
%%%
%%% Write your solution to problem 2 below.
%%% Uncomment the lines starting at the
%%% \newpage.
%%%
%%%%%%%%%%%%%%%%%%%%%%%%%%%%%%%%%%%%%%%%%%%%%

% \newpage
% \paragraph{Part 2 Solution}
% 

\end{problem}

\newpage
\begin{problem}{A Tiny Calculator (0 Points)}
For this exercise, we will build a calculator in \emph{reverse polish notation}. Also called \emph{postfix} notation, RPN is a mathematical notation in which every operator follows all of its operands. This notation is often used in stack-based and concatenative programming languages.

Here are some examples of expressions in infix and prefix notation:

\begin{tabular}{|l|l|}
\hline
\textbf{Infix} & \textbf{RPN}\\
\hline
\hline
$1 + 2$ & $1~2 +$\\
$3 - 4 + 5$ & $3~4 - 5 +$\\
$(3 - 4) * 5$ & $3~4~5 * -$\\
\hline
\end{tabular}

The algorithm for RPN is as follows:
\begin{algorithm}
\While{there are input tokens left}{
  Read the next token from input\;
  \eIf{token is a value}{
    Push it onto the stack\;
  }{
    It is already known that the other takes $n$ arguments\;
    \eIf{there are fewer than $n$ values on the stack}{
      Raise an error\;
    }{
      Pop the top $n$ values from the stack\;
    }
    Evaluate the operator with the values as arguments\;
    Push the returned results back onto the stack\;
  }
\eIf{there is only one value in the stack}{
  That value is the result\;
}{
  Error: the user input has too many values\;
}
}
\caption{Reverse Polish Notation algorithm.}
\end{algorithm}

The skeleton code for this problem is in \texttt{3/calc.ml} and is based on a snippet from Rosetta Code.

\paragraph{Part 1 (0 points)}
Compile \texttt{calc.ml}:
\begin{verbatim}
ocamlc str.cma calc.ml -o calc
\end{verbatim}
You should get the \texttt{Unimplemented} error when you run \texttt{calc}.

\paragraph{Part 2 (0 points)}
Understand the structure of the code that we have written for you. What do the functions \texttt{print\_answer}, \texttt{rpn\_eval}, and \texttt{interp\_and\_show} do?

\paragraph{Part 3 (0 points)}
Fill in the bodies of \texttt{interp} and \texttt{binop} so that running \texttt{calc} prints the following output:
\begin{verbatim}
***
3 2 5 + -
Token	Action	Stack
3	push	3. 
2	push	3. 2. 
5	push	3. 2. 5. 
+	add	3. 7. 
-	subtr	-4. 
-4.
***
3 2 + 5 -
Token	Action	Stack
3	push	3. 
2	push	3. 2. 
+	add	5. 
5	push	5. 5. 
-	subtr	0. 
0.
***
2 3 11 + 5 - *
Token	Action	Stack
2	push	2. 
3	push	2. 3. 
11	push	2. 3. 11. 
+	add	2. 14. 
5	push	2. 14. 5. 
-	subtr	2. 9. 
*	mult	18. 
18.
***
9 5 3 + 2 4 ^ - +
Token	Action	Stack
9	push	9. 
5	push	9. 5. 
3	push	9. 5. 3. 
+	add	9. 8. 
2	push	9. 8. 2. 
4	push	9. 8. 2. 4. 
^	exp	9. 8. 16. 
-	subtr	9. -8. 
+	add	1. 
1.
\end{verbatim}

\end{problem}

\end{document}
